% This is a summary for 
\documentclass[a4paper,11pt]{amsart}

\usepackage{amsmath}
\usepackage{amsfonts}
\usepackage{amsthm}
\usepackage{amssymb}
\usepackage[english]{babel}
\usepackage[utf8]{inputenc}
\usepackage{graphicx}
\usepackage[margin=1.25in]{geometry}
\usepackage{xcolor}
\usepackage{setspace}
\onehalfspace
\usepackage{hyperref}
\hypersetup{
    pdftitle={COM 391 Summary},
    pdfauthor={Moritz M. Konarski},
    pdfsubject={Summary},
    pdfkeywords={COM;}{391;}{computer;}{graphics}
}
\renewcommand\UrlFont{\color{blue}\rmfamily}
\usepackage{listings}

\definecolor{codegreen}{rgb}{0,0.6,0}
\definecolor{codegray}{rgb}{0.5,0.5,0.5}
\definecolor{codepurple}{rgb}{0.58,0,0.82}
\definecolor{backcolour}{rgb}{0.95,0.95,0.92}

\lstdefinestyle{mystyle}{
    backgroundcolor=\color{backcolour},
    commentstyle=\color{codegreen},
    keywordstyle=\color{magenta},
    numberstyle=\tiny\color{codegray},
    stringstyle=\color{codepurple},
    basicstyle=\ttfamily\footnotesize,
    breakatwhitespace=false,
    breaklines=true,
    captionpos=b,
    keepspaces=true,
    numbers=left,
    numbersep=5pt,
    showspaces=false,
    showstringspaces=false,
    showtabs=false,
    tabsize=4,
    language=C++
}
\lstset{style=mystyle}

\begin{document}

\title{COM 391 Summary}
\author{Moritz M. Konarski}
\date{\today}

\maketitle
\tableofcontents

\section{Introduction}

Summary for the course Computer Graphics (COM 391) at the American University 
of Central Asia (\url{https://www.auca.kg}) in Bishkek, Kyrgyzstan for the 
spring semester of 2020. In this course we will write a 3D graphics engine in 
C++ using OpenGL.

\section{History}

In 1963 \textbf{Sketchpad} by Ivan Sutherland was one of the first CAD 
prototypes. In 1968 the \textbf{Mother of all demos} showed a cursor, graphical 
UI, real time collaboration on documents, and much more. The first
semi-realistic shading algorithm was developed by Bui Tuong \textbf{Phong} at
the University of Utah in 1973. In 1975 the \textbf{Utah teapot} was first used 
as a 3D rendering test model that has since been a popular in-joke. In 1973 the 
first computer with a window-based GUI by Xerox Alto shipped and in 1984 the 
first \textbf{Macintosh} shipped and \textbf{Tron}, the first heavily CGI based
movie, was released. \textbf{Photoshop} was developed for that Mac in 1985. In 
the same year the \textbf{IRIS SGI Graphics Workstation} was released. The 1995
movie \textbf{Toy Story} was the first entirely computer-animated feature film.
\textbf{3D graphics accelerators} were becoming a thing with ATI, Nvidia, and 
3dfx.

\section{GPU Architecture}

GPU dies are generally larger than CPU dies and they contain many more cores.
Individual GPU cores are smaller than CPU cores though. In a CPU each core
generally has its own L1 and L2 cache and then all cores share the L3 cache. In
GPUs a cluster of cores shares L1 cache and clusters of clusters share L2
cache. Some CPU manufacturers are Intel, AMD, Qualcomm, IBM, and ARM. Some GPU
manufacturers are Nvidia, AMD, Qualcomm, Intel, and ARM.

\section{GPU APIs}

Khronos Group maintains a number of Graphics \textbf{APIs} (application
programming interfaces, a set of functions that allow the access to, in this 
case, GPU features). Those APIs are \textbf{OpenGL} (1, 2, 4, 4), OpenGL ES 
(2, 3), and Vulkan. Microsoft maintains \textbf{DirectX} (Direct3D), Apple 
maintains \textbf{Metal}, and AMD \textbf{Mantle}. Other APIs include CUDA, 
OpenCL, and OpenVG. While C, C++, Java etc are common CPU languages, HLSL, 
GLSL, and CUDA are common GPU languages. The programming languages used for
APIs are generally C/C++ but other languages like Javascript, C\#, Java, and 
Python can also be used. 

Popular libraries used with these APIs are \textbf{SDL2} (Single DirectMedia 
Layer, a low level library for accessing OpenGL or Direct3D), \textbf{GLM} 
(OpenGL Mathematics, a library providing mathematical tools like vectors and 
matrices as well as their operations), and \textbf{GLEW} (OpenGL Extension 
Wrangler library is needed to determine the supported OpenGL features on a 
platform at runtime). Some graphics engines include Unity, Unreal Engine, 
CRYENGINE, Frostbite. Graphics engines and APIs run on basically all systems, 
be it PC, console, phone, TV or embedded. VR and AR are also current 
development areas.

\section{Mathematical Basics}

\subsection{Vectors}

Basically just an ordered list of numbers. The \textbf{dimension} of a vector
is the number of elements in it. A vector has a magnitude (length) and
a direction while a \textbf{scalar} only has a magnitude. In game development
the most common vectors are 2D, 3D, and 4D vectors. Line or row vectors can be 
written like $\vec v = [1,2,3]$. Writing the same vector vertically gives
a column vector. Scalar values are generally represented by lowercase letters
(Greek or Roman) in italics: \textit{a, b, c,} $\alpha$, $\gamma$, $\theta$
while vectors are represented by bold lowercase letters \textbf{a, b, c, v} or
by an arrow over top $\vec a$, $\vec v$. To refer to elements of a vector,
subscript notation is used $\vec v_1$, $\vec v_2$ or $\vec v_x$, $\vec v_y$

Geometrically speaking a vector is an arrow that has a magnitude (length) and 
direction. The magnitude of a vector is always positive. The arrow part of
a drawn vector is called head, the end tail. The numeric and graphic
representations of vectors are identical. Vectors do not have a position in
some space, they only encode displacements of things relative to other things. 
The \textbf{zero vector} is a special vector which has 0s in all positions. Its
magnitude is 0 and it is the only vector where this is the case. It is also the
only vector that does not have a position. Because vectors describe 
displacements they can encode relative positions. If displacements are 
considered from the origin, they can be used to represent points.

\subsubsection{Mathematical Operations}

To \textbf{negate a vector} all of its elements need to be negated. This has 
the effect of flipping the direction of the vector.  

\textbf{Multiplying a vector} by a scalar results in all of its elements being 
multiplied by the scalar. This means the vector is stretched or compressed.  

To \textbf{add or subtract two vectors} from each other, their corresponding 
elements are added or subtracted. Geometrically that means putting the tail of 
one vector at the head of the other one. 

To find a \textbf{displacement vector between two points} A and B one subtracts 
A from B: $\vec d = B - A$.

The \textbf{magnitude or length of a vector} is found by taking the square root 
of the sum of the squares of all its elements $||\vec a|| = \sqrt{x^2 + y^2 + 
z^2}$.

\textbf{Unit vectors} (normalized vectors) encode only the direction of a 
vector and have the length 1 $\vec v_0 = \vec v / ||\vec v||$. 

The \textbf{dot product} is a product of two vectors where the corresponding 
elements are multiplied together and the results then added together to get one 
number. $\vec a * \vec b = \vec a_1 * b_1 + \vec a_2 * b_2 + \vec a_3 * b_3$. 
It also has the definition $\vec a * \vec b = ||\vec a|| * ||\vec b|| * 
\cos{\theta}$, $\theta$ being the angle between the vectors. From we know that 
if the two vectors are perpendicular to each other, their dot product must be 
0. 

The \textbf{cross product} can only be applied in at least 3D, is not 
commutative and yields another vector. Specifically, $\vec a \times \vec b$ 
yields a vector that is perpendicular to both original vectors. $\vec a \times 
\vec b = [a_2b_3-a_3b_2, a_3b_1-a_1b_3,a_1b_2-a_2b_1]$. This has the further
property that $||\vec a \times \vec b|| = ||\vec a||*||\vec b||*\sin{\theta}$.

\subsection{Matrices}

Matrices are really important in linear algebra and they generally describe the
relationships between two coordinate spaces. This is done by defining
a computation that transforms vectors from one coordinate space to another.
Matrices are rectangular grids that are made up of rows (horizontal) and
columns (vertical). Matrices are defined by the number of rows and columns that
they have, \textbf{n} rows by \textbf{m} columns as $n \times m$. Matrices are
represented by bold capital letters \textbf{M, A, R}. To refer to specific
elements in a matrix subscripts are used, first row (i) then column (j) as
$m_{ij}$.
\begin{equation}
    M = 
    \begin{bmatrix}\nonumber
        m_{11} & m_{12} & m_{13}\\
        m_{21} & m_{22} & m_{23}\\
        m_{31} & m_{32} & m_{33}
    \end{bmatrix}
\end{equation}

\textbf{Square matrices} are matrices that have the same number of rows and
columns. The most common examples are 2D, 3D, and 4D examples. 

\textbf{Diagonal elements} of a matrix are those where the number for row and
column are the same $m_{ii}$. These elements for the diagonal of the matrix.
All other elements are called non-diagonal.

A \textbf{diagonal matrix} has 0 in all but the diagonal. If the diagonal only
contains 1s the matrix is called \textbf{identity matrix} \textbf{I}. The
dimension is indicated by a subscript
\begin{equation}
    I_{3} = 
    \begin{bmatrix}\nonumber
        1 & 0 & 0\\
        0 & 1 & 0\\
        0 & 0 & 1
    \end{bmatrix}
\end{equation}
The identity matrix is important because it is the \textbf{multiplicative
identity} element for matrices meaning that multiplying by it does not change
a matrix $A \times I = I \times A = A$. Matrices with dimensions $1 \times m$ or 
$n \times 1$ can be seen as row or column vectors.

A square matrix can describe any linear transformation (transformation that
preserves straight and parallel lines and has no translation). This means these
matrices encode rotations and multiplications (stretching). A list of useful
transformations include
\begin{itemize}
\item Rotation: matrix that rotates coordinate spaces or vectors. For example
    rotating then counterclockwise by $\theta$ by finding $R\vec v$
    \begin{equation}
    R = 
    \begin{bmatrix}\nonumber
        \cos{\theta} & -\sin{\theta}\\
        \sin{\theta} & \cos{\theta}
    \end{bmatrix}.
    \end{equation}
\item Scale: increase or decrease lengths of a vector or distances by
    multiplying $S \vec v$
    \begin{equation}
    S = 
    \begin{bmatrix}\nonumber
        s_x & 0\\
        0 & s_y
    \end{bmatrix}.
    \end{equation}
\item Orthographic Projection: a form of parallel projection that represents 3D
    objects in 2D space. A simple example is the projection onto the plane
    $z=0$
    \begin{equation}
    P = 
    \begin{bmatrix}\nonumber
        1 & 0 & 0\\
        0 & 1 & 0\\
        0 & 0 & 0
    \end{bmatrix}.
    \end{equation}
\item Reflection: reflects and may also rotate at the same time. Example: $\pi
    / 2$ reflection
    \begin{equation}
    R = 
    \begin{bmatrix}\nonumber
        -1 & 0\\
        0 & 1
    \end{bmatrix}.
    \end{equation}
\end{itemize}

Multiplying a matrix by one of the base vectors $i,j,k$ yields the row
corresponding to the base vector used. In other words, the first row of
\textbf{M} contains the transformation of $i$ and so on. We can thus interpret
the rows of a matrix as the basis vectors of some coordinate system. If we
understand how the basis vectors are changed by the rows of a matrix we can
understand what the transformation does. Accordingly, to transform a vector
from its original coordinate system to a new one we multiply it by the
corresponding matrix. Thus a linear transformation is performed.

\subsubsection{Mathematical Operations}

\textbf{Matrix Transposition} is the process of flipping a matrix diagonally,
so it turns $M_{nm} \rightarrow M^T_{mn}$. For vectors a transposition turns
a row vector into a column vector and vice versa. The notation can also be used
to write column vectors in text like $[1,2,3]^T$. Double transpositions yield
the original matrix. All diagonal matrices are equal to their transposed forms
$D^T = D$.

\textbf{Scalar multiplication} for matrices $k*M$ multiplies all elements of
$M$ by $k$.

\textbf{Matrix multiplication} can be performed when the number of columns of
the first matrix is equal to the number of rows of the second matrix.
Multiplying a $r \times n$ matrix by a $n \times c$ matrix yields a $r \times
c$ matrix. If the number of columns does not equal the number of rows the
operation is not defined. Matrix multiplication is not commutative. It is
defined for $AB$ as each element $c_{ij}$ of the result matrix as the dot product of
row $i$ of \textbf{A} with column $j$ of \textbf{B}. If any matrix \textbf{M}
is multiplied by a square matrix S of the right size ($SM$ or $MS$) the
resulting matrix has the same dimension as \textbf{M}. Matrix multiplication is
not commutative but associative $(AB)C=A(BC)$. The transposition of a product
is the same as the product of transpositions with order reversed
$(AB)^T=B^TA^T$.

\textbf{Multiplying a vector and matrix} is possible if the number of columns
equals the number of rows, both operations yield vectors. Row vector times
matrix gives row vector, matrix times column vector gives column vector. Vector
multiplication distributes over vector addition $(\vec v + \vec w)M = \vec
v M + \vec w M$

\section{Book Notes}

\subsection{Chapter 1}

\subsubsection{What is 3D Math?}
\begin{itemize}
    \item math behind the geometry of a 3D world
    \item solving geometric problems algorithmically
    \item book uses C++
    \item basically all the basics one needs for computations in 3D
\end{itemize}

\subsubsection{Why You Should Read This Book}
\begin{itemize}
    \item an introductory book
    \item a toolbox that one can come back to
    \item it has math, geometry, and code in one
    \item good graphics that illustrate 3D maths and concepts
    \item accompanying website with demos and stuff \url{www.gamemath.com}
\end{itemize}

\subsubsection{Chapter Overview}
\begin{itemize}
    \item 1. Introduction
    \item 2. Cartesian Coordinates
    \item 3. Examples of Coordinate Spaces and Nesting
    \item 4. Vectors
    \item 5. Vector Operations
    \item 6. C++ Vector Class
    \item 10. Angular Displacement
\end{itemize}

\subsection{Chapter 2}

\subsubsection{1D Mathematics}
\begin{itemize}
    \item natural numbers / counting numbers, number line
    \item integers, negative numbers
    \item rational numbers, irrational numbers, real numbers
    \item complex numbers
    \item binary numbers and C++ number types
    \item \textbf{First Law of Computer Graphics}: If it looks right, it
        \textit{is} right.
\end{itemize}

\subsubsection{2D Cartesian Mathematics}
\begin{itemize}
    \item cartesian = rectangular
    \item centered around the arbitrary origin
    \item infinite in all directions, continuous division of intervals
    \item normal orientation: positive y up, positive x right
    \item all of those orientations can be transformed into each other
\end{itemize}

\subsubsection{From 2D to 3D}
\begin{itemize}
    \item 3D tends to be a lot more complicated than 2D
    \item we have 3 perpendicular axes 
    \item right handed is right hand with thumb pointed in +x, left handed same
        with thumb into +x
\end{itemize}

\subsection{Chapter 3}

\subsubsection{Why Multiple Coordinate Spaces?}
\begin{itemize}
    \item using more than one coordinate space tends to be easier even though
        one big world would be sufficient
    \item in some circumstances it makes sense to do certain things in separate
        coordinate systems and then bring them together when needed
\end{itemize}

\subsubsection{Some Useful Coordinate Spaces}
\begin{itemize}
    \item \textbf{World Space}: is the ultimate reference frame for all other 
        coordinate systems; the largest coordinate system we care about; also 
        know as \textbf{global} or \textbf{universal}; \textit{position, 
        orientation of objects (including camera), terrain elements, movement 
        with respect to world space}
    \item \textbf{Object Space}: associated with each individual object; this
        object space is moved and rotated with the object when it moves and
        rotates in world space; for example \textit{forward} is a concept of
        object space while \textit{east} would be world space; \textit{are
        other objects near, where is it in relation to me (front, left)}
    \item \textbf{Camera Space}: coordinate space associated with an observer;
        3D space that is a special object space where the camera defines the
        viewpoint of the scene; this defines what \textit{up} means with
        respect to the camera and not the world as a whole; \textit{is a point
        in front of the camera, is a point on screen or off screen, where on
        the screen is it, objects in relation to each other}
    \item \textbf{Inertial Space}: coordinate space is half-way between world
        space and object space; origin is the same as of origin space; axes of
        this space are parallel to world space; this basically takes the
        \textit{rotation} step out of the object to world space transition; 
        leaves only the \textit{translation} from world origin to object space origin
\end{itemize}

\subsubsection{Nested Coordinate Spaces}
\begin{itemize}
    \item an object is defined relative to the origin of object space
    \item the position and orientation of an object in world space is important
        for interactions with other objects
    \item the object's position can be figured out by positioning the origin in 
        world space and then figuring out relative positions
    \item \textbf{world space is the parent of object space}
    \item objects can have child spaces and those can have child spaces 
        themselves, this means coordinate spaces are nested
\end{itemize}

\subsubsection{Specifying Coordinate Spaces}
\begin{itemize}
    \item we need to specify an origin to specify the position
    \item the orientation is described by the coordinate axes
\end{itemize}

\subsubsection{Coordinate Space Transformations}
\begin{itemize}
    \item when we want to have two objects interact we need to express things
        in one coordinate system in terms of another one, e.g. a robot picking
        up an object where we know the position of the robot's hand in robot
        coordinates but need them in world coordinates
    \item we need to transform the position from world space to object space or
        vice versa -- the transformation does not actually move anything, it
        just expresses it in different coordinates
    \item first we rotate the object axes to get them parallel to the world
        axes
    \item then we translate the inertial space to line it up with the world
        space
\end{itemize}

\subsection{Chapter 4}

\subsubsection{Vector - Mathematical Definition}
See previous section on vectors.
\subsubsection{Vector - Geometric Definition}
See previous section on vectors.
\begin{itemize}
    \item displacement and velocity are vector values while distance and speed
        are scalar quantities
\end{itemize}

\subsection{Chapter 5}

\subsubsection{Linear Algebra}
See previous section on vectors.
\subsubsection{Typeface Conventions}
See previous section on vectors.
\subsubsection{Zero Vector}
See previous section on vectors.
\subsubsection{Negating a Vector}
See previous section on vectors.
\subsubsection{Linear Algebra}
See previous section on vectors.
\subsubsection{Vector Magnitude}
See previous section on vectors.
\subsubsection{Scalar Multiplication}
See previous section on vectors.
\subsubsection{Normalized Vectors}
See previous section on vectors.
\subsubsection{Vector Addition and Subtraction}
See previous section on vectors.
\subsubsection{Distance Formula}
See previous section on vectors.
\subsubsection{Vector Dot Product}
See previous section on vectors.
\subsubsection{Vector Cross Product}
See previous section on vectors.
\subsubsection{Linear Algebra Identities}
See previous section on vectors.

\subsection{Chapter 6}

\subsubsection{Matrix - Mathematical Definition}
See previous section on matrices.
\subsubsection{Matrix - Geometric Definition}
See previous section on matrices.
\begin{itemize}
    \item 
\end{itemize}

A chapter that might be nice to read to become familiar with how to effectively
use C++.

\subsection{Chapter 7}


\subsection{Chapter 10}

\end{document}
